% ======================================================================
% Antenna Analysis Report - 145 MHz Dipole
% ======================================================================

\documentclass[12pt,a4paper]{report}

\usepackage{graphicx}
\usepackage{geometry}
\usepackage{setspace}
\usepackage{pdfpages}
\usepackage{titlesec}
\usepackage{fancyhdr}
\usepackage{hyperref}
\usepackage{amsmath}

% ---------------- PAGE SETUP ----------------
\geometry{margin=1in}
\setstretch{1.2}
\pagestyle{fancy}
\fancyhf{}
\rhead{Antenna Analysis Report}
\lhead{145 MHz Dipole Antenna}
\cfoot{\thepage}
\hypersetup{
    colorlinks=true,
    linkcolor=blue,
    urlcolor=cyan
}

% ---------------- DOCUMENT START ----------------
\begin{document}

\begin{titlepage}
    \centering
    \vspace*{3cm}
    {\Huge \textbf{Antenna Analysis and Simulation Report}}\\[1cm]
    {\LARGE Dipole Antenna Design and Analysis at 145 MHz}\\[2cm]
    {\Large \textbf{Author:} Lakshya Varshney}\\[0.5cm]
    {\large B.Tech. (Electronics), COEP Technological University}\\[1cm]
    {\large Date: \today}\\[3cm]
    \vfill
\end{titlepage}

% ---------------- ABSTRACT ----------------
\section*{Theory and Discussion}
Antennas act as the interface between electrical circuits and the free-space electromagnetic field. 
They convert alternating current signals into propagating radio waves during transmission and vice versa during reception. 
The study of antenna parameters such as gain, impedance, and Voltage Standing Wave Ratio (VSWR) provides quantitative insight into how efficiently this energy conversion takes place. 

\section*{1. Why These Parameters Matter}
Each antenna parameter serves a specific purpose in ensuring reliable communication:
\begin{itemize}
    \item \textbf{Resonance and Impedance:} For efficient energy transfer, the antenna’s input impedance must match that of the transmission line—commonly 50~$\Omega$. A mismatch causes reflections, reducing transmitted power.
    \item \textbf{VSWR (Voltage Standing Wave Ratio):} VSWR quantifies mismatch. A perfect match yields VSWR = 1. Values up to 1.5 indicate good practical matching, where over 95\% of power is radiated.
    \item \textbf{Gain and Directivity:} Gain represents how effectively the antenna focuses radiated energy in a particular direction compared to an isotropic radiator. A half-wave dipole exhibits approximately 2.15~dBi gain, providing a reference for more complex directional antennas.
\end{itemize}

\section*{2. Theoretical Foundation}
The wavelength ($\lambda$) corresponding to a frequency ($f$) is derived from the speed of light ($c$) as:
\[
\lambda = \frac{c}{f}
\]
For 145~MHz, 
\[
\lambda = \frac{3\times10^8}{145\times10^6} = 1.943~\text{m}.
\]
A resonant half-wave dipole therefore has a total length:
\[
L = \frac{\lambda}{2} = 0.971~\text{m},
\]
which ensures that the current distribution along the arms produces maximum radiation efficiency and a purely resistive feed impedance.

At resonance, the antenna behaves like a resistive load of approximately 75~$\Omega$. 
When connected to a 50~$\Omega$ feedline, the reflection coefficient ($\Gamma$) is given by:
\[
\Gamma = \frac{Z_L - Z_0}{Z_L + Z_0} = \frac{75 - 50}{75 + 50} = 0.2,
\]
and the corresponding VSWR is:
\[
\text{VSWR} = \frac{1 + |\Gamma|}{1 - |\Gamma|} = \frac{1.2}{0.8} = 1.5.
\]
This moderate mismatch reflects only about 4\% of the transmitted power, making the antenna efficient for practical use.

\section*{3. Conceptual Understanding}
The MATLAB analysis provided a visual and quantitative verification of these theoretical predictions:
\begin{itemize}
    \item The impedance plot confirmed that the imaginary component crosses zero near 145~MHz, indicating resonance.
    \item The return loss ($S_{11}$) curve showed a deep null, corresponding to low reflection and good matching.
    \item The VSWR curve reached its minimum near 145~MHz, validating the impedance match.
    \item The radiation pattern displayed the classic toroidal (donut-shaped) structure of a dipole, with maximum radiation broadside to the element and nulls along its axis.
    \item The simulated gain remained near 2.15~dBi.
\end{itemize}

\section*{4. Broader Understanding}
This exploration revealed how antenna design is a balance between theory, simulation, and practical adjustment. 
Theoretical equations predict dimensions and behavior, while tools such as MATLAB visualize real-world performance including losses, bandwidth, and matching efficiency.  
Mastering these principles is essential for advanced applications such as satellite communication, where antenna performance directly affects link reliability and signal strength.

\newpage

% =====================================================
%               FIGURES SECTION (8 pages)
% =====================================================
\includepdf[pages={1},scale=0.9, pagecommand={
\vspace*{-2cm} % Pull figure up closer to the title
\section*{Simulation Results and Discussion}
\section*{Figure 1: 3D Geometry of the Dipole Antenna}
\noindent\textbf{Description:} The designed 145~MHz dipole consists of two arms, each approximately 0.517~m long. 
This corresponds to a total length of half the wavelength (1.034~m). 
The figure shows the 3D geometry and the feed point at the center.
\vspace{1cm}
}, addtotoc={1,section,1,{Simulation Results and Discussion},figures}]{Dipole_Simulation_Figures.pdf}


\includepdf[pages={2},scale=0.9,
pagecommand={
\section*{Figure 2: Input Impedance vs Frequency}
\noindent\textbf{Description:} The impedance plot shows the real and imaginary parts across the 130--160~MHz range. 
At 145~MHz, the imaginary component crosses zero, confirming resonance. 
The real component near 73~$\Omega$ represents the radiation resistance.
\vspace{1cm}
}]{Dipole_Simulation_Figures.pdf}

\includepdf[pages={3},scale=0.9,
pagecommand={
\section*{Figure 3: Return Loss (S11) vs Frequency}
\noindent\textbf{Description:} The S11 curve displays a deep dip around 145~MHz with return loss better than --15~dB, 
indicating a strong impedance match and minimal reflection.
\vspace{1cm}
}]{Dipole_Simulation_Figures.pdf}

\includepdf[pages={4},scale=0.9,
pagecommand={
\section*{Figure 4: VSWR vs Frequency}
\noindent\textbf{Description:} The VSWR plot shows a minimum value near 1.5 at resonance. 
This suggests that approximately 96\% of the transmitter power is radiated, with only 4\% reflected.
\vspace{1cm}
}]{Dipole_Simulation_Figures.pdf}

\includepdf[pages={6},scale=0.9,
pagecommand={
\section*{Figure 5: 3D Radiation Pattern}
\noindent\textbf{Description:} The 3D radiation pattern exhibits the characteristic toroidal shape of a dipole antenna. 
Maximum radiation occurs broadside to the dipole, with deep nulls along its axis.
\vspace{1cm}
}]{Dipole_Simulation_Figures.pdf}

\includepdf[pages={8},scale=0.9,
pagecommand={
\section*{Figure 6: Azimuth Pattern (Horizontal Plane)}
\noindent\textbf{Description:} The azimuth pattern demonstrates nearly uniform radiation in the horizontal plane (omnidirectional behavior). 
This makes the dipole ideal for ground-based VHF communication.
\vspace{1cm}
}]{Dipole_Simulation_Figures.pdf}

\includepdf[pages={7},scale=0.9,
pagecommand={
\section*{Figure 7: Elevation Pattern (Vertical Plane)}
\noindent\textbf{Description:} The elevation cut reveals two main lobes perpendicular to the dipole axis. 
The nulls at the top and bottom correspond to the ends of the dipole, where no current flows.
\vspace{1cm}
}]{Dipole_Simulation_Figures.pdf}

\includepdf[pages={5},scale=0.9,
pagecommand={
\section*{Figure 8: Gain (Directivity) vs Frequency}
\noindent\textbf{Description:} The gain (directivity) curve peaks at approximately 2.15~dBi near 145~MHz. 
This value aligns with the theoretical directivity of an ideal half-wave dipole, confirming simulation accuracy.
\vspace{1cm}
}]{Dipole_Simulation_Figures.pdf}
% =====================================================
%                   CONCLUSION
% =====================================================

\section*{Conclusion}
This project combined theoretical antenna fundamentals with MATLAB simulations to fully characterize a 145~MHz half-wave dipole.  
The simulated impedance and radiation behavior closely matched textbook predictions, validating both the mathematical and practical design.  
Such a dipole, due to its simplicity, balanced impedance, and moderate gain, remains a core element in VHF satellite communication systems.  
This work serves as a solid foundation for future experimental antenna construction and testing at the Institute’s Satellite Club.

% =====================================================
%                   REFERENCES
% =====================================================

\section*{References}
\begin{enumerate}
    \item Shubhendu Joardar, \textit{Basics of Antennas}, NIT Calicut, IIT Madras.
    \item MATLAB Antenna Toolbox Documentation, MathWorks, 2025.
    \item Youtube Videos
\end{enumerate}

\end{document}
